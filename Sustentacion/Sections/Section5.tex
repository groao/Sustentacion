%-------------------------------------------------------
\section{Conclusiones - Recomendaciones}
%-------------------------------------------------------
\begin{frame}{Conclusiones}
%-------------------------------------------------------
  \begin{block}{}
   \begin{itemize}
   \justifying
    \item<1->  El protocolo recurre a la auto-organización para que la diseminación de paquetes de información sea más eficiente y robusta frente a las dinámicas inherentes de las MANETs.
    \item<1->  Los mecanismos bio-inspirados constituyen una opción más eficiente en términos económicos y de complejidad.
    \item<1-> La cantidad máxima de dispositivos que podrán estar asociados a un propagador estará limitado por los recursos computacionales disponibles. Se corroboró que como factor de diseño un $\gamma =0.6$ máximo.\\
    \item<1->  Experimentalmente se evidenció que bajo ciertas condiciones presenta una reducción de Overhead de hasta el 24\% con respecto a IPv6.\\
    \item<1->  Delegar tareas de procesamiento y gestión de red hacia los dispositivos ubicados en el "borde" de las redes representa una alternativa efectiva para el despliegue a del IoT, reduciendo complejidad en la red, y blindando a las subredes de posibles vulnerabilidades de seguridad.
    \end{itemize}
  \end{block}
\end{frame}
%-------------------------------------------------------
\begin{frame}{Recomendaciones}
%-------------------------------------------------------
  \begin{block}{}
   \begin{itemize}
   \justifying
    \item<1->  Este protocolo de direccionamiento puede tener versiones adaptativas emulando efectos vistos en la decodificación del ADN, donde las moléculas poseen fragmentos de información que son activados y sintetizados sólo en ciertas condiciones.
    \item<1->  En aplicaciones reales, la limitación en la cantidad de terminales que pueden ser gestionados por los propagadores estará limitado por los recursos computacionales propios más no por el espacio de direccionamiento. En este sentido es viable sacrificar direccionamiento para modificar el espacio dedicado a los identificadores externos con el fin de hacer cabida a nuevos dispositivos y aplicaciones.
    \item<1-> En el entorno del IoT, los dispositivos de borde podrían aprovechar la infraestructura existente como mecanismo de transporte, es el caso de protocolos como powerline communication. Este tipo de re-usos tendría a su vez un impacto económico muy positivo que aumentaría la masificación del IoT.  
    \end{itemize}
  \end{block}
\end{frame}