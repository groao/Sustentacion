%-------------------------------------------------------
\section{Conclusiones - Recomendaciones}
%-------------------------------------------------------
\begin{frame}{Conclusiones}
%-------------------------------------------------------
  \begin{block}{Conclusiones}
   \begin{itemize}
   \justifying
    \item<1->  El protocolo recurre a la auto-organización para que la diseminación de paquetes de información sea más eficiente y robusta frente a las dinámicas inherentes de las MANETs.
    \item<1->  Los mecanismos bio-inspirados constituyen, en principio, una opción más eficiente en términos económicos y de complejidad.
    \item<1->  El modelo propuesto reduciría en gran proporción la señalización con respecto a IPv6 ajustándose apropiadamente a las características de las comunicaciones máquina-máquina.
    \item<1->  Delegar tareas de procesamiento y gestión de red hacia los dispositivos ubicados en el "borde" de las redes se plantea como una alternativa efectiva para el despliegue a gran escala del IoT, reduciendo complejidad a la red, mejorando significativamente los tiempos de latencia y blindando a las subredes de posibles vulnerabilidades de seguridad.
    \end{itemize}
  \end{block}
\end{frame}
%-------------------------------------------------------
\begin{frame}{Recomendaciones}
%-------------------------------------------------------
  \begin{block}{Recomendaciones}
   \begin{itemize}
   \justifying
    \item<1->  Este protocolo de direccionamiento puede tener versiones adaptativas emulando efectos vistos en la decodificación del ADN, donde las moléculas poseen fragmentos de información que son activados y sintetizados sólo en ciertas condiciones.
    \item<1->  En aplicaciones reales, la limitación en la cantidad de terminales que pueden ser gestionados por los propagadores estará limitado por los recursos computacionales propios más no por el espacio de direccionamiento. En este sentido es viable sacrificar direccionamiento para modificar el espacio dedicado a los identificadores externos con el fin de hacer cabida a nuevos dispositivos y aplicaciones.
    \item<1-> En el entorno del IoT, los dispositivos de borde podrían aprovechar la infraestructura existente como mecanismo de transporte, es el caso de protocolos como powerline communication. Este tipo de re-usos tendría a su vez un impacto económico muy positivo que aumentaría la masificación del IoT.  
    \end{itemize}
  \end{block}
\end{frame}